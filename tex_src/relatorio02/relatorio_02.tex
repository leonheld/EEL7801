\documentclass[11pt,a4paper]{report}
\usepackage[utf8]{inputenc}
\usepackage{amsmath}
\usepackage{amsfonts}
\usepackage{amssymb}
\usepackage{graphicx}
\usepackage{minted}

\title{Relatório 2  \\
	Projeto em Eletrônica I - EEL7801 \\ \vfill
	\normalsize{Universidade Federal de Santa Catarina - UFSC \\
		Professora: Daniela Ota Hisayasu Suzuki}
	\author{
		{Luiz Augusto Frazatto Fernandes: \it{17202752}} \\
		{Leonardo José Held: \it{17203984}}
}
}
\date{6 de Junho de 2019}
\begin{document}
	\maketitle
\chapter{Implementação do Algoritmo em C}
	\section{Modulação}

	\section{Demodulação}

\chapter{Comunicação entre os MCUs via cabo}
	
	\section{}
	
\chapter{Implementação dos Transdutores}
		https://ieeexplore.ieee.org/document/7993643
		https://www.edaboard.com/showthread.php?232682-Detecting-a-specific-audio-frequency
		https://forum.arduino.cc/index.php?topic=484397.0
		https://forum.arduino.cc/index.php?topic=541584.0
		
\chapter{Projeto das PCBs}
	As placas de circuito impresso (PCIs) usadas no projeto serão/foram projetadas através do software Eagle e produzidas no Laboratório de Montagem Mecatrônica (LMM), do Departamento de Automação e Sistemas da UFSC.
	software utilizado
	processo de criação
	referências
	
	\section{Simulação dos circuitos}
	programa utilizado

	\section{Processo de fabricação}
	especificações
	comentários práticos da implementação do circuito
	
\end{document}